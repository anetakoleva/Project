%%\label{section:Discussion}
%TODO check if this should be in the previous section
An initial objective of this project was to identify the main challenge that needs to be resolved in order to tackle the WSC.
From studying the existing approaches, we recognized the need for having a formal representation of a specific background knowledge during the reasoning process in the knowledge-based approaches. To this end, we proposed a categorization of the sentences for which similar knowledge is required for sentences belonging in the same category. We assume that for the different categories there can be different formalized rules which can be used during the reasoning process. 


The identified categories differ from the ones presented by Sharma and Baral \cite{2018CommonsenseKT} in the way they are defined. We rely on the context of the WS sentences, whereas in \cite{2018CommonsenseKT} the categorization is based on the structure of the sentence. Namely, we analyzed the entire WSC corpus and identified what is the commonsense knowledge that is needed for answering the question and from which category it would be. In Sharma and Baral \cite{2018CommonsenseKT} the categories are identified using a more general approach that is, they all have the same structure \textit{X prevents/follows/causes Y}. 

One unanticipated finding was that the rule-based Reasoning Algorithm given in Sharma and Baral \cite{2018CommonsenseKT} does not work as expected.  What came as a surprise was that the correct answer was retrieved even when \textit{the key rule} was removed. The key rule is the one which formalizes the knowledge type of the WSs.
As explained in Sharma and Baral \cite{2018CommonsenseKT}, for the given example this rule is what characterizes the knowledge type \textit{Property prevents Action}. In the example given in the supplementary document this rule is encoded as: \textbf{has\_k(weak\_1,prevents,lifts\_5)}. 
Intrigued by this result, we analyzed more closely the Reasoning Algorithm. What we discovered is that the answer returned depends on whether in the background knowledge there is information about the \textit{agent} or the \textit{recipient} of the action. 

Since the Reasoning Algorithm is well encoded, we decided to change only the encoding of the background knowledge so it would capture the characteristics to one of our identified categories. 
Having these rule in the background knowledge and then running the Reasoning Algorithm returns the correct answer. In contrast to the rule from the example in Sharma and Baral \cite{2018CommonsenseKT}, when the rule that formalizes the physical trait is removed from the background knowledge, no answer is retrieved. 

A good semantic graph representation of the WS is of high importance for correctly formalizing the available knowledge. To understand better the choice of \textit{agent} and the \textit{recipient} for the WS sentences, we analyzed more closely the work of the KParser. Although in most of the sentences that we parsed a correct graph representation was returned, in some of them there were many inconsistency. For example consider the following sentence:
\begin{itemize}
	\item[\textbf{S:}] \textbf{Joan made sure to thank Susan for all the help she had given.}
\end{itemize}

When parsed with the KParser, the semantic graph representation of this sentence consists of three disconnected graphs. Moreover, in the case when there are two consequent sentences in one WS the result from the KParser is too complicated for analyzing, let alone for formalizing in ASP. 



