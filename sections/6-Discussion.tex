\section{Discussion}
\label{section:Discussion}

An initial objective of this project was to identify the main challenge that needs to be resolved in order to tackle the WSC. From studying the existing approaches, we recognized the need for having a formal representation of a specific background knowledge during the reasoning process in the knowledge based approaches. To this end, we proposed a categorization of the sentences for which similar knowledge is required for sentences belonging in the same category.

\begin{comment}
Emphasize these two
1. our categories are content based -> a combination of categories and knowledge about them could be a future work
2. we tried to identify the least amount of necessary commonsense knowledge (due to efficiency), which can consist of various different knowledge(s), that can exclude each other
\end{comment}

The identified categories differ from the ones presented by Sharma and Baral \cite{2018CommonsenseKT} in the way they are defined. We rely on the content of the WS sentences, whereas in \cite{2018CommonsenseKT} the categorization is based on the structure of the sentence. Namely, we analyzed the entire WSC corpus and we tried to identify the least amount of necessary commonsense knowledge that is needed for answering the question and from which category it would be.  In Sharma and Baral \cite{2018CommonsenseKT} the categories are identified using a more general approach that is, they all have the same structure \textit{X prevents/follows/causes Y}. 

One unanticipated finding was that the rule-based RA presented in Sharma and Baral \cite{2018CommonsenseKT} does not work as expected. What came as a surprise was that the correct answer was retrieved even when rule describing the knowledge type was removed. Intrigued by this result, we analyzed more closely the reasoning algorithm. What we discovered is that the answer returned depends on whether in the background knowledge there is information about the \textit{agent} or the \textit{recipient} of the action. With the intention to improve the reasoning procedure by  Sharma and Baral \cite{2018CommonsenseKT}, we decided to change only the encoding of the background knowledge so it would capture the characteristics to one of our identified categories. Having these rules as the background knowledge and then running the RA returns the correct answer. In contrast to the rule from the example in Sharma and Baral \cite{2018CommonsenseKT}, when the rule that formalizes the physical trait is removed from the background knowledge, no answer is retrieved. 

A good semantic graph representation of the WS is of high importance for correctly formalizing the available knowledge. To understand better the choice of \textit{agent} and the \textit{recipient} for the WS sentences, we analyzed more closely the work of the KParser. Although in most of the sentences that we parsed a correct graph representation was returned, in some of them there were many inconsistencies. For example consider the following sentence (part of WS \#3):\\ 
\begin{itemize}
	\item[\textbf{S:}] \textbf{Joan made sure to thank Susan for all the help she had given.}
\end{itemize}
Figure \ref{Disc} shows the semantic graph representation of this sentence which consists of three disconnected graphs. Moreover, in the case when there are two consequent sentences in one WS the result from the KParser is too complicated for analyzing, let alone for formalizing in ASP. 

\begin{figure} 
	\centering
	%\documentclass{standalone}
%\usepackage{tikz}
%\usetikzlibrary{arrows, shapes, trees, positioning}

%\begin{document}
	
	\tikzset{
		treenode/.style = {shape=rectangle, rounded corners,
			draw, align=center,
			top color=white, bottom color=blue!20},
		root/.style     = {treenode, font=\ttfamily\normalsize},
		env/.style      = {treenode, font=\ttfamily\normalsize},
		dummy/.style    = {circle,draw},
		level 1/.style={sibling distance=2cm, level distance = 7em},
		level 2/.style={sibling distance=2cm,level distance = 5em}, 
		level 3/.style={sibling distance=4cm},
		blueRed/.style={env, top color=blue, bottom color=red} 
	}
	
	\begin{tikzpicture}
	[
	grow                    = down,
	%sibling distance        = 50em,
	level distance          = 5em,
	edge from parent/.style = {draw, -latex},
	every node/.style       = {font=\footnotesize},
	sloped
	]
	
	
	
	% Graph 1
	\node[root]   (root)   {\underline{made-2}}
		child {node [env] (s1) {\textbf{Joan-1}}
		child {node [env] {\textit{person}}
			edge from parent node[above] {inst\_of}}
		edge from parent node[above] {agent}}
		child {node [env] {\textbf{sure-3}}
			edge from parent node [above] {object}}
		child {node [env] {\textit{make}}
			edge from parent node [above] {inst\_of}} ;
	
	
	
	% Graph 2
	\node[root]   (lift)   [right=35mm of root] {\underline{thank-5}}
	child {node [root] (s1) {\textbf{Susan-6}}
		child {node [env] {\textit{person}}
			edge from parent node[above] {inst\_of}}
		child {node [env] {\textbf{help-10}}
			child {node [env] {\textbf{all-8}}
				edge from parent node[above] {identifier}}
			edge from parent node[above] {\tiny{destination}}}	
		edge from parent node[above] {recipient}}
	child {node[env] (S4) {\textit{thank}}
		edge from parent node[above] {inst\_of}};
	
	% Graph 3
	\node[root]   (lift)   [right=35mm of lift] {\underline{given-13}}
	child {node [root] (s1) {\underline{had-12}}
		child {node [env] {have}
			edge from parent node [above] {inst\_of}}
		edge from parent node[above] {support\_verb}}
	child {node [root] (s1) {\textbf{she-11}}
		child {node [env] {\textit{\textit{person}}}
			edge from parent node [above] {inst\_of}}
		edge from parent node[above] {agent}}
	child {node[env] (S4) {\textit{give}}
		edge from parent node[above] {inst\_of}};
	
	
	
	\end{tikzpicture}
%\end{document}
	\caption{\label{Disc} Graph representation for \textbf{S}.}
\end{figure}
