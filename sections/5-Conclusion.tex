%\section{Conclusion}
%\label{section:Conclusion}

\paragraph{Conclusion}
In this report we presented a detail analysis of the WSC. We introduced the main components of a WS and identified the need for additional, background knowledge as a main challenge for correctly resolving one. We presented an extensive literature review on the existing different proposals and their contributions towards solving the WSC. Looking into the different approaches, we highlighted that neither the Machine learning based nor the Knowledge-based approaches alone are sufficient for achieving high accuracy on the WSC. Therefore, with the intention to improve the process of extracting and formalizing relevant knowledge, we analyzed in more details the WSC problems and identified different categories of WSs. The categories that we presented can enhance the work on both the  Machine learning and the Knowledge-based approaches or can be used as a bridge between them. We found the process of identification of the categories and deciding the right category for the WSs to be quite challenging. It is not surprising that so far too little attention has been paid to the content of the WSs. In addition, we used an existing reasoning algorithm, which when adapted as for one of the identified categories retrieves correct answers for WSs. 
It is important to note that our work presented here is an analysis of the WSC and not an attempt towards solving it. Moreover, due to the limited number of WSC problems, the presented categories were identified in a ``backward-engineering" manner, that is by considering directly the test set. 

\paragraph{Future Work}
To support the assumption that identifying the category of a WS can improve the extraction of relevant additional knowledge, the characteristics of the meta-concepts from each category ought to be formalized. Analyzing the semantic graphs for the WS sentences could be a good starting point for recognizing these characteristics. Afterwards, these formalized concepts need to be implemented and tested in an appropriate reasoning algorithm. 
Another interesting extension would be to explore an implementation of knowledge-enhanced neural networks \cite{DBLP:conf/aaai/MaPC18} with commonsense knowledge from the different categories and test their performance. 
Finally, since the existing semantic parsers often provide incomplete and disconnected graphs, it would be worth considering different representations of the input Winograd sentence and question. 

%TODO organize these three paragraphs 
 Part of the aim of this project is to develop a hypothesis which can improve the process of extracting relevant knowledge for the reasoning process. Recently, as proposed by Ma et al. \cite{DBLP:conf/aaai/MaPC18}, applying knowledge injection during the training of deep neural networks can lead to improvement of the result of the neural network. Using commonsense knowledge databases, different neural networks can be pre-trained with knowledge from a specific category. After analyzing the Winograd input sentence, a network trained with the knowledge from the identified category can be used in the process of extracting relevant background knowledge. 
Additionally, formalized rules describing the characteristics of a category can be fed to neural networks before training \cite{DBLP:conf/aaai/RoychowdhuryDG18}. In this way, the rules would be a guidance for the network during the training phase. Not only could this speed up the training of the network by reducing the amount of required data, but might also support explanations for the predicted answer. An approach that exploits the advantage of formalized rules, capturing characteristics of different areas of background knowledge, can possibly merge together the strengths of both categories of approaches, Machine Learning and Knowledge-Based Systems. 
