%\section{Introduction}
\label{section:introduction}

The Winograd Schema Challenge (WSC) was proposed by Levesque et al. \cite{DBLP:conf/kr/LevesqueDM12} as a new test in Artificial Intelligence (AI) and possibly as an alternative to the Turing test. The test captures a difficult pronoun disambiguation problem which is an easy task for humans, but still remains an unsolved challenge for computers. The WSC is formulated in such manner that it requires intelligence and genuine understanding of real world situations for solving it. For this reason it is argued that a computer that is able to solve the WSC with human-like accuracy must be able to perform human-like thinking \cite{DBLP:conf/kr/LevesqueDM12}. 
Different types of commonsense knowledge and reasoning are required to solve the Winograd Schema problems. Thus, the WSC has been proposed as a method for testing automated commonsense reasoning. This idea of designing a  machine which could apply commonsense reasoning was first proposed by McCarthy \cite{McCarthy:1960:PCS:889202}. He recognized that commonsense reasoning is a trait of intelligence and therefore tried to express it in formal logic so that it can be used for building a truly intelligent machine. Since then, formalizing commonsense reasoning has been an open challenge in the AI field.   

Inspired by an example from Terry Winograd \cite{article}, the WSC corpus consists of sentences like the following:
\begin{itemize}
	\item[\textbf{S1:}] \textbf{\underline{The city councilmen} refused the demonstrators a permit because \underline{they} feared violence.}
	\item[\textbf{S2:}] \textbf{The city councilmen refused \underline{the demonstrators} a permit because \underline{they} advocated violence.}
\end{itemize}

The task in the WSC is to identify the correct referent of the pronoun \textit{they}. The difference between the sentences is the special word, in this case \textit{feared/advocated}. Depending on this word, the referent of the pronoun \textit{they} changes. In the first sentence it referes to \textit{the city councilmen} and in the second to \textit{the demonstrators}. This characteristic is what makes the WSC task a restricted form of the coreference resolution problem. The goal in the coreference resolution problem is to identify all the correct antecedents for a pronoun by relying on information about the gender and the number of the pronoun \cite{Coref}. In the WSC problems the candidates antecedents are always the same gender and number as the ambiguous pronoun, so this information is not sufficient for resolving the correct referent.
In order to identify the referent of the ambiguous pronoun, one needs to have knowledge about the relations between the nouns, the verb phrase and the special word from the Winograd sentence.

Various approaches have been suggested for solving the WSC.
More generally, the studies of tackling the WSC can be divided into two categories: Machine learning and Knowledge-based. 
The division is based on the main techniques applied for obtaining the correct answer. The first category contains the approaches which rely on machine learning and deep learning techniques \cite{DBLP:conf/emnlp/RahmanN12, DBLP:journals/corr/LiuJLZWH16,  DBLP:journals/corr/abs-1806-02847}. Indeed, recent approaches from this category \cite{DBLP:journals/corr/abs-1806-02847, radford2019language} are the most recent ones to perform the best, with reported accuracy of over 70\%. In the second category are the approaches which rely on knowledge-based systems \cite{DBLP:conf/aaai/SharmaB16,DBLP:conf/emnlp/EmamiCTSC18,DBLP:conf/kr/Schuller14}. These require to have formally represented knowledge and procedures that should be able to reason with that knowledge. Many of the previously proposed approaches \cite{DBLP:conf/ijcai/SharmaVAB15, DBLP:conf/kr/Schuller14, DBLP:journals/corr/LiuJLZWH16, DBLP:conf/emnlp/EmamiCTSC18} recognized that answering the WSC correctly requires additional knowledge to what is in the given sentence. However, to the best of our knowledge few approaches have analyzed the knowledge in the available WSC sentences so far. 

For humans, commonsense reasoning comes naturally  because of the available background knowledge and because of the understanding about the surrounding world that humans have. In order for machines to be able to appear as if they would do commonsense reasoning, a huge amount of non-domain specific knowledge is needed \cite{DBLP:journals/ibmsj/McCarthyMSGLMMRSS02}. To address this issue, there have been attempts to develop repositories of common knowledge such as Cyc \cite{DBLP:journals/cacm/Lenat95} and ConceptNet \cite{articleC}. However, it is unclear whether these knowledge bases can ever be completed or if they contain all the necessary information for commonsense reasoning. 

%The area of commonsense reasoning is divided into subareas, each addressing different domains of commonsense \cite{Davis:1990:RCK:83819}. So far, there have been significant improvements in the areas dealing with automated reasoning about taxonomic categories, about time and about actions and change \cite{DBLP:journals/cacm/DavisM15}.  
Motivated by the need for background knowledge we analyzed the sentences from the WSC corpus and identified six different categories of commonsense reasoning. We describe the process of annotating the WSC problems with these categories and we describe their characteristics. 
From the conducted survey of existing approaches, one promising proposal from the knowledge-based approaches was of particular interest to us. Namely, because of the similarity with our work and the recognized space for improvement, we additionally investigate the proposal by Sharma and Baral \cite{2018CommonsenseKT} more closely.


The rest of this report is structured as follows:
In chapter 2 we introduce the WSC and explore approaches from both categories which have made significant contributions towards solving the WSC. In chapter 3 we discuss in more details the approach by Sharma and Baral \cite{2018CommonsenseKT} and the proposed Reasoning Algorithm.
Chapter 4 presents the result of our analysis along with examples and description for the identified categories. In this chapter we also discuss the semantic graphs of the WSs which we analyzed thoroughly and formalized. Finally, chapter 5 concludes the paper and explores potential future work.


