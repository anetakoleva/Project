\documentclass[svgnames, final, 20pt]{beamer} % beamer 3.10: do NOT use
% option hyperref={pdfpagelabels=false} !
  %\documentclass[final,hyperref={pdfpagelabels=false}]{beamer} % beamer 3.07:
% get rid of beamDer warnings
  \mode<presentation> {  %% check  for examples
    \usetheme{Iccl}
  }
  \usepackage[english]{babel}
  \usepackage[latin1]{inputenc}
  \usepackage{amsmath,amsthm, amssymb, latexsym}

% \usepackage[table]{xcolor}
\usepackage{tikz,pgflibrarysnakes}
\usetikzlibrary{shapes,positioning,arrows,backgrounds,fit,calc}

 \usepackage{ragged2e} 
  %\usepackage{times}\usefonttheme{professionalfonts}  % times is obsolete
%  \usefonttheme[onlymath]{serif}
%  \boldmath
%  \usepackage[orientation=portrait,size=a0,scale=1,debug]{beamerposter}         
             % e.g. for DIN-A0 poster
 \usepackage[orientation=portrait,size=a0,scale=1.4,debug]{beamerposter}    
  %\usepackage[orientation=portrait,size=a1,scale=1.4,grid,debug]{beamerposter} 
                % e.g. for DIN-A1 poster, with optional grid and debug output
%  \usepackage[size=custom,width=90,height=200,scale=1,debug]{beamerposter}  
                  % e.g. for custom size poster
 
%\usepackage[orientation=portrait,size=a0,scale=1.0,printer=rwth-glossy-uv.df]{ beamerposter}   % e.g. for DIN-A0 poster with rwth-glossy-uv printer check
  % ...
  %
%%% make Bib Symbol larger
% \setbeamertemplate{bibliography item}[beamerposterarticle] 
\usepackage{booktabs}

\usepackage{epstopdf}
\usepackage{colortbl}
\usepackage{tabularx}
\usepackage{makecell}
\usepackage{listings}

\lstdefinestyle{SC}{
	language=erlang,
	xleftmargin=.2\textwidth,
	xrightmargin=.2\textwidth
}
\lstset{ 
	basicstyle=\scriptsize,
	showstringspaces=false,
	showspaces=false,
	commentstyle=\color{blue},
	keywordstyle=\bfseries\color{purple},
	captionpos=b,
	frame=lines
}

\usepackage{natbib}
\bibliographystyle{plain}


% \usepackage{paralist}
% \renewenvironment{thebibliography}[1]{%
%   \section*{\refname}\inparaenum[{[}1{]}]}{\endinparaenum}
% \renewcommand{\bibitem}[1]{\item}

\usepackage{url}
\usepackage{adjustbox}
\usepackage{tikz}
\definecolor{green}{rgb}{0,0.7,0}


\newenvironment{shrinkeq}[1]
{ \bgroup
  \addtolength\abovedisplayshortskip{#1}
  \addtolength\abovedisplayskip{#1}
  \addtolength\belowdisplayshortskip{#1}
  \addtolength\belowdisplayskip{#1}}
{\egroup\ignorespacesafterend}




\newcommand{\AB}{\mathit{Ab}}
\newcommand{\Ab}{\mathit{ab}}
\newcommand{\Add}{\mathit{add}}
\newcommand{\Al}{\mathit{al}}
\newcommand{\Alive}{\mathit{a}}
\newcommand{\ARSC}{\textsc{ARSC}}
\newcommand{\AR}{\longrightarrow_a}
\newcommand{\ARS}{\rightarrow_a}
\newcommand{\Arson}{\mathit{a}}

\newcommand{\Beach}{\mathit{beach}}
\newcommand{\Body}{\mathit{body}}
\newcommand{\beetle}{\mathit{beetle}}
\newcommand{\Broken}{\mathit{broken}}


% \newcommand{\CFndrybeach}{\Cond(\neg \DryLeaves, \Beach)}
% \newcommand{\CFnlightbeach}{\Cond(\neg \Lightning, \Beach)} 

\newcommand{\Fact}{\mathsf{facts}}


\newcommand{\CalA}{{\cal A}}
\newcommand{\CalC}{{\cal C}}
\newcommand{\CalD}{{\cal D}}
\newcommand{\CalE}{{\cal E}}
\newcommand{\CalExt}{\textsf{ext}}
\newcommand{\CalF}{{\cal F}}
\newcommand{\CalIC}{{\cal IC}}
\newcommand{\CalL}{{\cal L}}
\newcommand{\CalM}{{\cal M}}
\newcommand{\CalO}{{\cal O}}
\newcommand{\CalP}{{\cal P}}
\newcommand{\CalS}{{\cal S}}
\newcommand{\CalT}{{\cal T}}
\newcommand{\Cause}{\mathit{cause}}
\newcommand{\CFndryff}{\Cond(\neg \DryLeaves, \Beach)}
\newcommand{\CFnlightff}{\Cond(\neg \Lightning, \Beach)}
\newcommand{\Cig}{\mathit{cig}}
\newcommand{\Clouds}{\mathit{clouds}}
\newcommand{\comment}[1]{\marginpar[\raggedleft{}\hspace{10pt}\em #1]{\raggedright{}\hspace{10pt}\em #1}}
%\newcommand{\comment}[1]{}
\newcommand{\Cond}{\mathit{cond}}
\newcommand{\Core}{\textsc{core}}
\newcommand{\CR}{\longrightarrow_c}
\newcommand{\CRS}{\rightarrow_c}


\newcommand{\depends}{\mathit{dep}}
\newcommand{\Def}{\mathit{def}}
\newcommand{\Dead}{\mathit{d}}
\newcommand{\dodge}{\mathit{dodge}}
\newcommand{\Drop}{\mathit{drop}}
\newcommand{\DryLeaves}{\mathit{d}}

\newcommand{\Earlier}{\mathit{earlier\_than\_6am}}
\newcommand{\EExt}{\CalE_{\textsf{Ext}}}
\newcommand{\Execution}{\mathit{e}}

\newcommand{\ferrari}{\mathit{ferrari}}
\newcommand{\False}{\textit{false}}
\newcommand{\Forest}{\mathit{forest}}
\newcommand{\ForestFire}{\mathit{f}}
\newcommand{\FR}{\longrightarrow_f}

\newcommand{\Ground}{\mathit{g}}

\newcommand{\Head}{\mathit{head}}
\newcommand{\hummer}{\mathit{hummer}}
\newcommand{\hotsummer}{\mathit{hot\_summer}}

\newcommand{\IA}{\longrightarrow_{\mathit{ia}}}
\newcommand{\IC}{\mathit{IC}}
\newcommand{\Imprisoned}{\mathit{imprisoned}}
\newcommand{\Inex}{\mathit{inex}}
\newcommand{\Inside}{\mathit{inside}}
\newcommand{\Inspect}{\mathit{inspect}}
\newcommand{\IF}{\longrightarrow_{\mathit{if}}}
\newcommand{\IR}{\longrightarrow_{\mathit{ir}}}
\newcommand{\IT}{\longrightarrow_{\mathit{it}}}
\newcommand{\IU}{\longrightarrow_{\mathit{iu}}}
\newcommand{\IV}{\longrightarrow_{\mathit{iv}}}

\newcommand{\Jim}{\mathit{jim}}

\newcommand{\Ken}{\mathit{ken}}
\newcommand{\KD}{\mathit{k}}

\newcommand{\Level}{\ell}
\newcommand{\Lightning}{\ell}
\newcommand{\lit}{\mathsf{lit}}

\newcommand{\MaxRev}{\textsc{MaxRev}}
\newcommand{\MinRev}{\textsc{MinRev}}
\newcommand{\ModelsLMWC}{\models_{\mbox{\tiny \em
      \L}}^{\mathit{lmwc}}}
\newcommand{\Move}{\mathit{move}}
\newcommand{\MoveAD}{\mathit{move\_after\_death}}
\newcommand{\MRFA}{MRFA}
\newcommand{\Myua}{\!\uparrow\!}
%\newcommand{\Mylabel}[1]{\label{#1} \comment{#1}}
\newcommand{\Mylabel}[1]{\label{#1}}

\newcommand{\OS}{\mathit{os}}
\newcommand{\Outside}{\mathit{outside}}

\newcommand{\PRevLR}{\textsc{PRevLR}}
\newcommand{\PRevRL}{\textsc{PRevRL}}
\newcommand{\PExLR}{\textsc{PExLR}}
\newcommand{\PExRL}{\textsc{PExRL}}
\newcommand{\PExt}{{\CalP_{\textsf{ext}}}}
\newcommand{\porsche}{\mathit{porsche}}

\newcommand{\Rain}{\mathit{raining}}
\newcommand{\Raining}{\mathit{raining}}
\newcommand{\Rev}{\mathit{rev}}
\newcommand{\RIC}{\textsc{RIC}}
\newcommand{\RifleA}{\mathit{ra}}
\newcommand{\RifleB}{\mathit{rb}}
\newcommand{\RSC}{\textsc{RSC}}
\newcommand{\RR}{\longrightarrow_r}
\newcommand{\RRS}{\rightarrow_r}

\newcommand{\SA}{\longrightarrow_{\mathit{sa}}}
\newcommand{\SC}{\mathit{sc}}
\newcommand{\Signal}{\mathit{s}}
\newcommand{\SF}{\longrightarrow_{\mathit{sf}}}
\newcommand{\Shrink}{\vspace{-.99mm}}
\newcommand{\SI}{\longrightarrow_{\mathit{si}}}
\newcommand{\SR}{\longrightarrow_{\mathit{sr}}}
\newcommand{\ST}{\longrightarrow_{\mathit{st}}}
\newcommand{\SU}{\longrightarrow_{\mathit{su}}}
\newcommand{\SV}{\longrightarrow_{\mathit{sv}}}
\newcommand{\SES}{\mathit{ses}}

\newcommand{\breaks}{\mathit{breaks}}
\newcommand{\dropglass}{\mathit{dropglass}}

\newcommand{\Term}{\mathit{TR}}
\newcommand{\TR}{\longrightarrow_t}
\newcommand{\TrafficJam}{\mathit{traffic\_jam}} 
\newcommand{\True}{\textit{true}}

\newcommand{\udf}{\mbox{U}}
\newcommand{\Unknown}{\textit{unknown}}
\newcommand{\UR}{\longrightarrow_u}

\newcommand{\Vacuous}{\textit{vacuous}}

\newcommand{\WC}{\mathit{wc}}

\newcommand{\Zombie}{\mathit{zombie}}


\newcommand{\PhiS}{\Phi_{\mathit{S}}}


\newcommand{\atoms}{\mathsf{atoms}}

\newcommand{\Lfp}{\mathsf{lfp}\;}
\newcommand{\Psimodels}{\models_{s}}
\newcommand{\ud}{\mathsf{undef}}
\newcommand{\defined}{\mathsf{def}}


\newcommand{\ModelsWCS}{\models_{\mathit{wcs}}}

\renewcommand{\ModelsLMWC}{\models_{\mbox{\tiny \em
      \L}}^{\mathit{lmwc}}}


\newcommand{\LeastMod}{\mathsf{lm}}
\newcommand{\Luka}{{\mbox{\tiny\L}}}
\newcommand{\Lmwcl}{\LeastMod_{\Luka}\WC}


\newcommand{\MA}{\textsf{A}}
\newcommand{\MI}{\textsf{I}}
\newcommand{\ME}{\textsf{E}}
\newcommand{\MO}{\textsf{O}}


\newcommand{\oa}{{\color{myorange}\textit{a\ }}}
\newcommand{\ob}{{\color{myorange}\textit{b\ }}}
\newcommand{\oc}{{\color{myorange}\textit{c\ }}}
\definecolor{myorange}{HTML}{FF6400}
%%%%%%%%%

\title{On Commonsense Domains within  the Winograd Schema Challenge} 
\author{ Aneta Koleva \hfill
Supervisor: Emmanuelle-Anna Dietz \;}
\date      {Bertinoro 2019}

\definecolor{expbg}{rgb}{0.85,0.85,0.85}
\definecolor{progbg}{rgb}{0,0.188,0.372}


\setbeamercolor{expcol}{fg=black,bg=expbg}
\setbeamercolor{progcol}{fg=white,bg=headline2colorbg}


\newcommand{\EINS}{\hspace{5mm}}
\newcommand{\ZWEI}{\hspace{15mm}}

\newcommand{\betweensmallandtiny}{\fontsize{16}{18}\selectfont}


\renewcommand{\udf}{\mbox{U}}

\newcommand{\WSC}{WSC}
\newcommand{\WFS}{WFS}

\begin{document}

\begin{frame}[fragile]
\vskip -2ex

\begin{columns}[t]
\hskip 1ex
\column[t]{.49\textwidth}


\begin{block}{Commonsense Reasoning in Computers}
\justify
Hector Levesque (2011) proposes a new test for assessing computer intelligence that requires use of commonsense reasoning when predicting the correct answer.\medskip \\
\begin{beamercolorbox}[rounded=true,sep=0.3cm]{expcol}

\begin{itemize}
	\item[S:] \underline{The trophy} does not fit into \underline{the brown suitcase}\\ because {\color{myorange}it} is too {\color{myorange}[small/large]}.
	\item[Q:] What is too [small/large]?
	\item[A:] The suitcase/the trophy.
\end{itemize}	
\end{beamercolorbox}

\begin{figure}[width]
\begin{tikzpicture}[scale = 2]
\begin{scope}[blend group = soft light]
\fill[red!30!white]   ( 90:1.2) circle (2);
\fill[green!30!white] (210:1.2) circle (2);
\fill[blue!30!white]  (330:1.2) circle (2);
\end{scope}
\node at ( 90:2)    {NLP};
\node at ( 210:2)   {Reasoning};
\node at ( 330:2)   {Knowledge};
\node [font=\Large] {WSC};
\end{tikzpicture}
\end{figure}

\end{block}
 

\begin{block}{Winograd Schema Challenge (WSC)}
\justify
\begin{itemize}
	\item Structure of a Winograd Schema
		\begin{itemize}
		\item Sentence containing two nouns, \\one ambiguous {\color{myorange}pronoun} and a {\color{myorange}special word}.
		\item Question asking about the referent of the pronoun.
		\item Two possible answers corresponding to\\ the noun phrases in the sentence.
	%	\item The correct answer to the question changes when \\ the special word is switched
	\end{itemize}
	\item Characteristics
	 \begin{itemize}
		\item Easy to answer for an adult English speaker.
		\item Always contains {\color{myorange}special word}.
		\item \textit{Google-proof} - statistical methods over large text corpora should not be able to resolve a WS.
	\end{itemize}
\end{itemize}


\end{block}
\begin{block}{Machine-Learning vs Knowledge-Based Approaches}

\begin{figure}[h!]
	\centering
	\setlength{\tabcolsep}{25pt}
	\renewcommand{\arraystretch}{1}
	{\footnotesize
		\begin{tabularx}{\columnwidth}{ l| c c c c }
			\hline
			\makecell[c]{\textbf{Technique}}   &{\makecell{\textbf{PDPs}\\ Size \\  Correct}} &{\makecell{\textbf{WSC}\\Size  \\  Correct}} &{\makecell[l]{{\textbf{WSC*}}\\Size  \\  Correct}} &\makecell[c]{\textbf{Remarks}} \\ \hline 
			
			\makecell{Supervised ranking\\ SVM model \cite{RahmanN12}} & NA &\makecell{NA} & \makecell{282 - 30\% \\ 205 - 73\% } &{\makecell[l]{ -provided additional dataset set \\
					-no evaluation on WSC dataset  }}  \\\hline
			
			\makecell{Classification task\\with NN \cite{W18-4105}} & NA &{\makecell{  282 - 100\% \\ 157 - 56\%}} & \makecell{ 282 - 30\% \\ 177 - 63\%}  &\makecell[l]{-first to use substitution of the \\ pronoun with the antecedents}\\\hline
			
			\makecell{Knowledge Enhanced\\Embeddings (KEE) \cite{DBLP:journals/corr/LiuJLZWH16}}  &\makecell{ 60-100\% \\ 40 - 66.7\%}  & \makecell{NA} & NA &\makecell[l]{-best results in the 2016\\ WSC competition}\\\hline
			
			\makecell{Google's language\\ models  \cite{DBLP:journals/corr/abs-1806-02847}} &\makecell{ 60-100\% \\ 42 - 70\% }& \makecell{273 - 100\% \\  173 - 63.7\%} & NA & \makecell[l]{-no reasoning involved in the\\discovery of the correct answer \\-state-of-the-art for PDPs}\\\hline
			
			\makecell{OpenAI language\\ models \cite{radford2019language}} & NA &\makecell{273 - 100\% \\ 193 - 70.70\%} & NA &\makecell[l]{-current state-of-the-art for WSC\\ -requires a lot of data for training\\-results are not reproducible }\\ \Xhline{3\arrayrulewidth}
			
			\makecell{Graphs with \\Relevance theory \cite{Peter}} & NA  & \makecell{4 - 2.6\% \\ 4 - 100\%} & NA &\makecell[l]{-manual construction of graphs\\-first representation of WS\\ as dependency graph}\\\hline
			
			\makecell{2 identified \\categories \cite{DBLP:conf/ijcai/SharmaVAB15}} & NA  & \makecell{71 -25\% \\ 49 - 69\%} & NA &\makecell[l]{-first attempt of identifying\\commonsense knowledge types \\-developed the KParser} \\\hline
			
			\makecell{Semantic relations\\ categories \cite{2018CommonsenseKT}} & NA &\makecell{100 - 34\% \\ 100 - 100\%} &  \makecell{138 - 14\% \\ 111 - 80\%} &\makecell[l]{-provided Reasoning Algorithm\\ -identified 12 commonsense types\\ which capture the entire WSC}  \\\hline
			
			\makecell{Knowledge hunting\\ framework \cite{DBLP:conf/emnlp/EmamiCTSC18}}& NA & \makecell{273 - 100\% \\ 119 - 43.5\%} & NA & \makecell[l]{-refined query generation\\-developed an algorithm for \\scoring the retrieved sentences}\\\hline
			
		\end{tabularx}
	}
\end{figure}
\end{block}
\begin{block}{References}

\bibliographystyle{plain}
\tiny{
\bibliography{literatur}
}
\end{block}





\column{.49\textwidth}

\begin{block}{Knowledge Types Identification and Reasoning\\(Sharma and Baral, 2018)}
\vspace{0.8cm}
\begin{itemize}
	\item Identified 12 {\color{myorange}knowledge types} which cover the entire WSC dataset.
	\item The categorization is based on the structure of the Winograd sentence.
	\item 10 of the knowledge types are based on different interactions between entities, actions and properties 
	\item Developed a {\color{myorange}logical reasoning algorithm}. 
	\item Evaluated on 100 problems from WSC and achieved {\color{myorange}100\%} accuracy


\begin{beamercolorbox}[rounded=true,sep=0.3cm]{expcol}
Extracted knowledge: ``weak y prevents y lifts".\\
Knowledge type ``Property prevents Action"\\
ASP encoding:\\
\begin{lstlisting}[language = Prolog, style=SC, numbers=right,
numberstyle=\tiny]
has_k(weak_1,is_trait_of,y_2).
has_k(weak_1,instance_of,weak).
has_k(y_2,instance_of,entity).
has_k(weak_1,prevents,lifts_5).
has_k(lifts_5,instance_of,lift).
has_k(lifts_5,agent,y_2).
\end{lstlisting}
\end{beamercolorbox}
\item The reasoning algorithm {\color{myorange}does not consider} rule 4 in the reasoning process.
\end{itemize}

\begin{block}{Categorization of Winograd Schemas}
\justify 
\begin{itemize}
	\item Inductively analyzed the WSC dataset and identified {\color{myorange}6 categories}.
	\item The categorization is based on the {\color{myorange}content} of the Winograd sentence.
	\item Two annotators annotated the entire WSC corpus with these categories.
	\item Calculated Cohen's kappa - measure for inter-rater agreement $\kappa=0.66$
\end{itemize}
 
\vspace{1cm}
\resizebox{\textwidth}{!}{	
	\begin{tabular}{  l | l  }
		\hline
		\textbf{Category}  & \textbf{Example} \\ \hline
		1. Physical &\textbf{S:} John couldn't see the stage with Billy in front of him because he is so {\color{myorange}[short/tall]}.\\&\textbf{Q:} Who is so [short/tall]? \\\hline
		
		2. Emotional &\textbf{S:}  Frank felt {\color{myorange}[vindicated/crushed]} when his longtime rival Bill\\&revealed that he was the winner of the competition.\\&\textbf{Q:} Who was the winner of the competition?\\\hline
		
		3. Interactions &\textbf{S:} Joan made sure to thank Susan for all the help she had {\color{myorange}[given/received]}. \\&\textbf{Q:} Who had [given/received] help?\\ \hline
		
		4. Comparison &\textbf{S:} Joe's uncle can still beat him at tennis, even though he is 30 years {\color{myorange}[older/younger]}.\\&\textbf{Q:} Who is [older/younger]?\\ \hline
		
		5. Causal &\textbf{S:} Pete envies Martin {\color{myorange}[because/although]} he is very successful.\\&\textbf{Q:} Who is very successful? \\ \hline 
		
		6. Multiple knowledge &\textbf{S:}  		
		Sam and Amy are passionately in love, but Amy's parents are unhappy about it,\\& because they are {\color{myorange}[snobs/fifteen]}.  \\&\textbf{Q:} Who are [snobs/fifteen]?\\ \hline
	\end{tabular} } 

\begin{block} {Winograd Schemas from the Physical Category}
\textbf{The trophy doesn't fit into the brown suitcase because it's too small.} 
\vspace{1cm}
\begin{figure}[width=\columnwidth]
	\centering
	\tikzset{
		treenode/.style = {shape=rectangle, rounded corners,
			draw, align=center,
			top color=white, bottom color=blue!20},
		root/.style     = {treenode, font=\ttfamily\normalsize, bottom color=red!30},
		env/.style      = {treenode, font=\ttfamily\normalsize},
		dummy/.style    = {circle,draw},
		level 1/.style={sibling distance=18cm, level distance = 3em},
		level 2/.style={sibling distance=7cm,level distance = 5em},
		blueRed/.style={env, top color=blue, bottom color=red} 
	}
	%\begin{document}
	\begin{tikzpicture}
	[
	grow                    = down,
	sibling distance        = 50em,
	level distance          = 5em,
	edge from parent/.style = {draw, -latex},
	every node/.style       = {font=\footnotesize},
	sloped
	]
	\node [root] {is-3}
	child { node [env, blue] {it-11}
		child {node [env] {small-14}
			edge from parent node [below] {trait} }
		edge from parent node [above] {agent}}
	child { node [env, bottom color=red!30] {fit-5} 
		child {node [env]  {does-3}
			edge from parent node [above] {support verb} }	
		child {node [env, blue]  {{\small{trophy-2}}}
			edge from parent node [above] {agent} }
		child {node [env, blue]  {suitcase-9}
			edge from parent node [above] {recipient}}
		child {node [env]  {not-4}
			edge from parent node [above] {negation} }
		edge from parent node [above] {cause}};
	
	\end{tikzpicture}
\end{figure}

\textbf{The man couldn't lift his son because he was so weak.} \\
\begin{figure}
	\centering
	\tikzset{
		treenode/.style = {shape=rectangle, rounded corners,
			draw, align=center,
			top color=white, bottom color=blue!20},
		root/.style     = {treenode, font=\ttfamily\normalsize, bottom color=red!30},
		env/.style      = {treenode, font=\ttfamily\normalsize},
		dummy/.style    = {circle,draw},
		level 1/.style={sibling distance=18cm, level distance = 3em},
		level 2/.style={sibling distance=7cm,level distance = 5em},
		blueRed/.style={env, top color=blue, bottom color=red} 
	}
	%\begin{document}
	\begin{tikzpicture}
[
grow                    = down,
sibling distance        = 50em,
level distance          = 5em,
edge from parent/.style = {draw, -latex},
every node/.style       = {font=\footnotesize},
sloped
]
\node [root] {was-10}
child { node [env, blue] {he-9}
	child {node [env] {weak-11}
		edge from parent node [below] {trait} }
	edge from parent node [above] {agent}}
child { node [env, bottom color=red!30] {lift-5} 
	child {node [env]  {could-3}
		edge from parent node [above] {modal verb} }	
	child {node [env, blue]  {man-2}
		edge from parent node [above] {agent} }
	child {node [env, blue]  {son-7}
		edge from parent node [above] {recipient}}
	child {node [env]  {not-4}
		edge from parent node [above] {negation} }
	edge from parent node [above] {cause}};
	\end{tikzpicture}
\end{figure}

\end{block}
\vspace{-3cm}
\end{block}
\end{block}

\begin{block}{Reasoning algorithm}
Change of the formalization of the background knowledge such that it contributes to the reasoning procedure.
\begin{beamercolorbox}[rounded=true,sep=0.3cm]{expcol}
\begin{lstlisting}[language = Prolog,  caption={Additional knowledge},label=code2,numbers=left,
numberstyle=\tiny]
has_k(small_1,is_trait_of,y_2) :- has_k(fits_5, recipient, y_2), 
				not has_k(fits_5,modifier,could_3).
has_k(y_2,instance_of,entity).
has_k(fits_5,recipient,y_2).
\end{lstlisting}

\end{beamercolorbox}
\end{block}
\end{columns}

\end{frame}
\end{document}
  
