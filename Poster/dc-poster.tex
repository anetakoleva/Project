\documentclass[svgnames, final, 20pt]{beamer} % beamer 3.10: do NOT use
% option hyperref={pdfpagelabels=false} !
  %\documentclass[final,hyperref={pdfpagelabels=false}]{beamer} % beamer 3.07:
% get rid of beamDer warnings
  \mode<presentation> {  %% check  for examples
    \usetheme{Iccl}
  }
  \usepackage[english]{babel}
  \usepackage[latin1]{inputenc}
  \usepackage{amsmath,amsthm, amssymb, latexsym}

% \usepackage[table]{xcolor}
\usepackage{tikz,pgflibrarysnakes}
\usetikzlibrary{shapes,positioning,arrows,backgrounds,fit,calc}

 \usepackage{ragged2e} 
  %\usepackage{times}\usefonttheme{professionalfonts}  % times is obsolete
%  \usefonttheme[onlymath]{serif}
%  \boldmath
%  \usepackage[orientation=portrait,size=a0,scale=1,debug]{beamerposter}         
             % e.g. for DIN-A0 poster
 \usepackage[orientation=portrait,size=a0,scale=1.4,debug]{beamerposter}    
  %\usepackage[orientation=portrait,size=a1,scale=1.4,grid,debug]{beamerposter} 
                % e.g. for DIN-A1 poster, with optional grid and debug output
%  \usepackage[size=custom,width=90,height=200,scale=1,debug]{beamerposter}  
                  % e.g. for custom size poster
 
%\usepackage[orientation=portrait,size=a0,scale=1.0,printer=rwth-glossy-uv.df]{ beamerposter}   % e.g. for DIN-A0 poster with rwth-glossy-uv printer check
  % ...
  %
%%% make Bib Symbol larger
% \setbeamertemplate{bibliography item}[beamerposterarticle] 
\usepackage{booktabs}

\usepackage{epstopdf}
\usepackage{colortbl}
\usepackage{tabularx}

\usepackage{makecell}
\usepackage{listings}

\lstdefinestyle{SC}{
	language=erlang,
	xleftmargin=.2\textwidth,
	xrightmargin=.2\textwidth
}
\lstset{ 
	basicstyle=\scriptsize,
	showstringspaces=false,
	showspaces=false,
	commentstyle=\color{blue},
	keywordstyle=\bfseries\color{purple},
	captionpos=b,
	frame=lines
}

\usepackage{natbib}
\bibliographystyle{plain}


% \usepackage{paralist}
% \renewenvironment{thebibliography}[1]{%
%   \section*{\refname}\inparaenum[{[}1{]}]}{\endinparaenum}
% \renewcommand{\bibitem}[1]{\item}

\usepackage{url}
\usepackage{adjustbox}
\usepackage{tikz}
\definecolor{green}{rgb}{0,0.7,0}


\newenvironment{shrinkeq}[1]
{ \bgroup
  \addtolength\abovedisplayshortskip{#1}
  \addtolength\abovedisplayskip{#1}
  \addtolength\belowdisplayshortskip{#1}
  \addtolength\belowdisplayskip{#1}}
{\egroup\ignorespacesafterend}




\newcommand{\AB}{\mathit{Ab}}
\newcommand{\Ab}{\mathit{ab}}
\newcommand{\Add}{\mathit{add}}
\newcommand{\Al}{\mathit{al}}
\newcommand{\Alive}{\mathit{a}}
\newcommand{\ARSC}{\textsc{ARSC}}
\newcommand{\AR}{\longrightarrow_a}
\newcommand{\ARS}{\rightarrow_a}
\newcommand{\Arson}{\mathit{a}}

\newcommand{\Beach}{\mathit{beach}}
\newcommand{\Body}{\mathit{body}}
\newcommand{\beetle}{\mathit{beetle}}
\newcommand{\Broken}{\mathit{broken}}


% \newcommand{\CFndrybeach}{\Cond(\neg \DryLeaves, \Beach)}
% \newcommand{\CFnlightbeach}{\Cond(\neg \Lightning, \Beach)} 

\newcommand{\Fact}{\mathsf{facts}}


\newcommand{\CalA}{{\cal A}}
\newcommand{\CalC}{{\cal C}}
\newcommand{\CalD}{{\cal D}}
\newcommand{\CalE}{{\cal E}}
\newcommand{\CalExt}{\textsf{ext}}
\newcommand{\CalF}{{\cal F}}
\newcommand{\CalIC}{{\cal IC}}
\newcommand{\CalL}{{\cal L}}
\newcommand{\CalM}{{\cal M}}
\newcommand{\CalO}{{\cal O}}
\newcommand{\CalP}{{\cal P}}
\newcommand{\CalS}{{\cal S}}
\newcommand{\CalT}{{\cal T}}
\newcommand{\Cause}{\mathit{cause}}
\newcommand{\CFndryff}{\Cond(\neg \DryLeaves, \Beach)}
\newcommand{\CFnlightff}{\Cond(\neg \Lightning, \Beach)}
\newcommand{\Cig}{\mathit{cig}}
\newcommand{\Clouds}{\mathit{clouds}}
\newcommand{\comment}[1]{\marginpar[\raggedleft{}\hspace{10pt}\em #1]{\raggedright{}\hspace{10pt}\em #1}}
%\newcommand{\comment}[1]{}
\newcommand{\Cond}{\mathit{cond}}
\newcommand{\Core}{\textsc{core}}
\newcommand{\CR}{\longrightarrow_c}
\newcommand{\CRS}{\rightarrow_c}


\newcommand{\depends}{\mathit{dep}}
\newcommand{\Def}{\mathit{def}}
\newcommand{\Dead}{\mathit{d}}
\newcommand{\dodge}{\mathit{dodge}}
\newcommand{\Drop}{\mathit{drop}}
\newcommand{\DryLeaves}{\mathit{d}}

\newcommand{\Earlier}{\mathit{earlier\_than\_6am}}
\newcommand{\EExt}{\CalE_{\textsf{Ext}}}
\newcommand{\Execution}{\mathit{e}}

\newcommand{\ferrari}{\mathit{ferrari}}
\newcommand{\False}{\textit{false}}
\newcommand{\Forest}{\mathit{forest}}
\newcommand{\ForestFire}{\mathit{f}}
\newcommand{\FR}{\longrightarrow_f}

\newcommand{\Ground}{\mathit{g}}

\newcommand{\Head}{\mathit{head}}
\newcommand{\hummer}{\mathit{hummer}}
\newcommand{\hotsummer}{\mathit{hot\_summer}}

\newcommand{\IA}{\longrightarrow_{\mathit{ia}}}
\newcommand{\IC}{\mathit{IC}}
\newcommand{\Imprisoned}{\mathit{imprisoned}}
\newcommand{\Inex}{\mathit{inex}}
\newcommand{\Inside}{\mathit{inside}}
\newcommand{\Inspect}{\mathit{inspect}}
\newcommand{\IF}{\longrightarrow_{\mathit{if}}}
\newcommand{\IR}{\longrightarrow_{\mathit{ir}}}
\newcommand{\IT}{\longrightarrow_{\mathit{it}}}
\newcommand{\IU}{\longrightarrow_{\mathit{iu}}}
\newcommand{\IV}{\longrightarrow_{\mathit{iv}}}

\newcommand{\Jim}{\mathit{jim}}

\newcommand{\Ken}{\mathit{ken}}
\newcommand{\KD}{\mathit{k}}

\newcommand{\Level}{\ell}
\newcommand{\Lightning}{\ell}
\newcommand{\lit}{\mathsf{lit}}

\newcommand{\MaxRev}{\textsc{MaxRev}}
\newcommand{\MinRev}{\textsc{MinRev}}
\newcommand{\ModelsLMWC}{\models_{\mbox{\tiny \em
      \L}}^{\mathit{lmwc}}}
\newcommand{\Move}{\mathit{move}}
\newcommand{\MoveAD}{\mathit{move\_after\_death}}
\newcommand{\MRFA}{MRFA}
\newcommand{\Myua}{\!\uparrow\!}
%\newcommand{\Mylabel}[1]{\label{#1} \comment{#1}}
\newcommand{\Mylabel}[1]{\label{#1}}

\newcommand{\OS}{\mathit{os}}
\newcommand{\Outside}{\mathit{outside}}

\newcommand{\PRevLR}{\textsc{PRevLR}}
\newcommand{\PRevRL}{\textsc{PRevRL}}
\newcommand{\PExLR}{\textsc{PExLR}}
\newcommand{\PExRL}{\textsc{PExRL}}
\newcommand{\PExt}{{\CalP_{\textsf{ext}}}}
\newcommand{\porsche}{\mathit{porsche}}

\newcommand{\Rain}{\mathit{raining}}
\newcommand{\Raining}{\mathit{raining}}
\newcommand{\Rev}{\mathit{rev}}
\newcommand{\RIC}{\textsc{RIC}}
\newcommand{\RifleA}{\mathit{ra}}
\newcommand{\RifleB}{\mathit{rb}}
\newcommand{\RSC}{\textsc{RSC}}
\newcommand{\RR}{\longrightarrow_r}
\newcommand{\RRS}{\rightarrow_r}

\newcommand{\SA}{\longrightarrow_{\mathit{sa}}}
\newcommand{\SC}{\mathit{sc}}
\newcommand{\Signal}{\mathit{s}}
\newcommand{\SF}{\longrightarrow_{\mathit{sf}}}
\newcommand{\Shrink}{\vspace{-.99mm}}
\newcommand{\SI}{\longrightarrow_{\mathit{si}}}
\newcommand{\SR}{\longrightarrow_{\mathit{sr}}}
\newcommand{\ST}{\longrightarrow_{\mathit{st}}}
\newcommand{\SU}{\longrightarrow_{\mathit{su}}}
\newcommand{\SV}{\longrightarrow_{\mathit{sv}}}
\newcommand{\SES}{\mathit{ses}}

\newcommand{\breaks}{\mathit{breaks}}
\newcommand{\dropglass}{\mathit{dropglass}}

\newcommand{\Term}{\mathit{TR}}
\newcommand{\TR}{\longrightarrow_t}
\newcommand{\TrafficJam}{\mathit{traffic\_jam}} 
\newcommand{\True}{\textit{true}}

\newcommand{\udf}{\mbox{U}}
\newcommand{\Unknown}{\textit{unknown}}
\newcommand{\UR}{\longrightarrow_u}

\newcommand{\Vacuous}{\textit{vacuous}}

\newcommand{\WC}{\mathit{wc}}

\newcommand{\Zombie}{\mathit{zombie}}


\newcommand{\PhiS}{\Phi_{\mathit{S}}}


\newcommand{\atoms}{\mathsf{atoms}}

\newcommand{\Lfp}{\mathsf{lfp}\;}
\newcommand{\Psimodels}{\models_{s}}
\newcommand{\ud}{\mathsf{undef}}
\newcommand{\defined}{\mathsf{def}}


\newcommand{\ModelsWCS}{\models_{\mathit{wcs}}}

\renewcommand{\ModelsLMWC}{\models_{\mbox{\tiny \em
      \L}}^{\mathit{lmwc}}}


\newcommand{\LeastMod}{\mathsf{lm}}
\newcommand{\Luka}{{\mbox{\tiny\L}}}
\newcommand{\Lmwcl}{\LeastMod_{\Luka}\WC}


\newcommand{\MA}{\textsf{A}}
\newcommand{\MI}{\textsf{I}}
\newcommand{\ME}{\textsf{E}}
\newcommand{\MO}{\textsf{O}}


\newcommand{\oa}{{\color{myorange}\textit{a\ }}}
\newcommand{\ob}{{\color{myorange}\textit{b\ }}}
\newcommand{\oc}{{\color{myorange}\textit{c\ }}}
\definecolor{myorange}{HTML}{FF6400}
%%%%%%%%%

\title{On Commonsense Domains within  the Winograd Schema Challenge} 
\author{ Aneta Koleva \hfill
Supervisor: Emmanuelle-Anna Dietz \;}
\date      {Bertinoro 2019}

\definecolor{expbg}{rgb}{0.85,0.85,0.85}
\definecolor{progbg}{rgb}{0,0.188,0.372}


\setbeamercolor{expcol}{fg=black,bg=expbg}
\setbeamercolor{progcol}{fg=white,bg=headline2colorbg}


\newcommand{\EINS}{\hspace{5mm}}
\newcommand{\ZWEI}{\hspace{15mm}}

\newcommand{\betweensmallandtiny}{\fontsize{16}{18}\selectfont}


\renewcommand{\udf}{\mbox{U}}

\newcommand{\WSC}{WSC}
\newcommand{\WFS}{WFS}

\begin{document}

\begin{frame}[fragile]
\vskip -2ex

\begin{columns}[t]
\hskip 1ex
\column[t]{.49\textwidth}


\begin{block}{Commonsense Reasoning in Computers}
	\justify
	\vspace{-1.2cm}
	Levesque (2011) proposes a new test for assessing computer intelligence that requires the use of commonsense reasoning.\medskip \\
	\begin{beamercolorbox}[rounded=true,sep=0.3cm]{expcol}
		
		\begin{itemize}
			\item[S:] \underline{The trophy} does not fit into \underline{the brown suitcase}\\ because {\color{myorange}it} is too {\color{myorange}[small/large]}.
			\item[Q:] What is too [small/large]?
			\item[A:] The suitcase/the trophy.
		\end{itemize}
	\end{beamercolorbox}
 Add (least necessary) background knowledge and apply to~reasoning algorithm.
	
	\begin{figure}[width]
		\begin{tikzpicture}[scale = 2]
		\begin{scope}[blend group = soft light]
		\fill[red!30!white]   ( 90:1.2) circle (2);
		\fill[green!30!white] (210:1.2) circle (2);
		\fill[blue!30!white]  (330:1.2) circle (2);
		\end{scope}
		\node at ( 90:2)    {NLP};
		\node at ( 210:2)   {Reasoning};
		\node at ( 330:2)   {Knowledge};
		\node [font=\Large] {WSC};
		\end{tikzpicture}
	\end{figure}
	
\end{block}
 
\vspace{-1.8cm}
\begin{block}{Winograd Schema Challenge (WSC)}
	\justify
	\begin{itemize}
		\item Structure of a Winograd Schema:
		\begin{center}
			\begin{tabular}{ l | c }
				Sentence containing {\color{myorange}two nouns} &  {\color{myorange}trophy}, {\color{myorange}suitcase} \\
				\hline 
				One ambiguous {\color{myorange}pronoun} & {\color{myorange}it} \\  
				\hline
				A {\color{myorange}special word} & {\color{myorange}small}/{\color{myorange} large}\\
				\hline
				{\color{myorange}Question} about the referent of the pronoun & What is too [{\color{myorange}small}/{\color{myorange}large}]\\
				\hline
				{\color{myorange}Two} possible {\color{myorange}answers} & {\color{myorange}The suitcase}/{\color{myorange}the trophy}
			\end{tabular}
		\end{center}
		\item Characteristics:
		\begin{itemize}
			\item Easy to answer for an adult English speaker.
			\item Always contains {\color{myorange}special word}.
			\item \textit{Google-proof} - statistical methods over large text corpora should not be able to resolve a WS.
		\end{itemize}
	\end{itemize}
	%	\item Structure of a Winograd Schema
	%		\begin{itemize}
	%		\item Sentence containing two nouns, \\one ambiguous {\color{myorange}pronoun} and a {\color{myorange}special word}.
	%		\item Question asking about the referent of the pronoun.
	%		\item Two possible answers corresponding to\\ the noun phrases in the sentence.
	%	%	\item The correct answer to the question changes when \\ the special word is switched
	%	\end{itemize}
	%	\item Characteristics
	%	 \begin{itemize}
	%		\item Easy to answer for an adult English speaker.
	%		\item Always contains {\color{myorange}special word}.
	%		\item \textit{Google-proof} - statistical methods over large text corpora should not be able to resolve a WS.
	%	\end{itemize}
	%\end{itemize}
	
	
\end{block}
\vspace{-0.5cm}
\begin{block}{Machine-Learning vs Knowledge-Based Approaches}
	\begin{figure}[h!]
		\centering
		\setlength{\tabcolsep}{25pt}
		\renewcommand{\arraystretch}{1}
		{\footnotesize
			\begin{tabularx}{\columnwidth}{ l | c | c | c | l }
				\hline
				\makecell[c]{\textbf{Technique}}   &{\makecell{\textbf{PDPs}\\ Size \\  Correct}} &{\makecell{\textbf{WSC}\\Size  \\  Correct}} &{\makecell{\textbf{WSC*}\\Size  \\  Correct}} &\makecell[c]{\textbf{Remarks}} \\ \hline 
				
				\makecell{Supervised ranking\\ SVM model \cite{RahmanN12}} & NA &\makecell{NA} & \makecell{282 - 30\% \\ 205 - 73\% } &{\makecell[l]{ -provided additional dataset set \\
						-no evaluation on WSC dataset  }}  \\\hline
				
				\makecell{Classification task\\with NN \cite{W18-4105}} &\makecell{NA}  &{\makecell{  282 - 100\% \\ 157 - 56\%}} & \makecell{ 282 - 30\% \\ 177 - 63\%}  &{\makecell[l]{-first to use substitution of the \\ pronoun with the antecedents}}\\\hline
				
				\makecell{Knowledge Enhanced\\Embeddings (KEE) \cite{DBLP:journals/corr/LiuJLZWH16}}  &\makecell{ 60-100\% \\ 40 - 66.7\%}  & \makecell{NA} & \makecell{NA} &\makecell[l]{-best results in the 2016\\ WSC competition}\\\hline
				
				\makecell{Google's language\\ models  \cite{DBLP:journals/corr/abs-1806-02847}} &\makecell{ 60-100\% \\ 42 - 70\% }& \makecell{273 - 100\% \\  173 - 63.7\%} & NA & \makecell[l]{-no reasoning involved in the\\discovery of the correct answer \\-state-of-the-art for PDPs}\\\hline
				
				\makecell{OpenAI language\\ models \cite{radford2019language}} & NA &\makecell{273 - 100\% \\ 193 - 70.70\%} & NA &\makecell[l]{-current state-of-the-art for WSC\\ -requires a lot of data for training\\-results are not reproducible }\\ \Xhline{3\arrayrulewidth}
				
				\makecell{Graphs with \\Relevance theory \cite{Peter}} & NA  & \makecell{4 - 2.6\% \\ 4 - 100\%} & NA &\makecell[l]{-manual construction of graphs\\-first representation of WS\\ as dependency graph}\\\hline
				
				\makecell{2 identified \\categories \cite{DBLP:conf/ijcai/SharmaVAB15}} & NA  & \makecell{71 -25\% \\ 49 - 69\%} & NA &\makecell[l]{-first attempt of identifying\\commonsense knowledge types \\-developed the KParser} \\\hline
				
				\makecell{Semantic relations\\ categories \cite{2018CommonsenseKT}} & NA &\makecell{100 - 34\% \\ 100 - 100\%} &  \makecell{138 - 14\% \\ 111 - 80\%} &\makecell[l]{-provided Reasoning Algorithm\\ -identified 12 commonsense types\\ which capture the entire WSC}  \\\hline
				
				\makecell{Knowledge hunting\\ framework \cite{DBLP:conf/emnlp/EmamiCTSC18}}& NA & \makecell{273 - 100\% \\ 119 - 43.5\%} & NA & \makecell[l]{-refined query generation\\-developed an algorithm for \\scoring the retrieved sentences}\\\hline
				
			\end{tabularx}
		}
	\end{figure}
	*Additional dataset with 943 WS provided in \cite{DBLP:conf/emnlp/RahmanN12} .
\end{block}
\vspace{-1cm}
\begin{block}{References}
	
	\bibliographystyle{plain}
	\tiny{
		\bibliography{literatur}
	}
\end{block}


\column{.49\textwidth}

\begin{block}{Knowledge Types Identification and Reasoning\\(Sharma and Baral, 2018)}

\begin{itemize}
	\item Identified 12 {\color{myorange}knowledge types} which cover entire WSC dataset.
	\item Categorization based on the {\color{myorange}structure} of the Winograd sentence.
	\item 10 knowledge types based on interaction between entities~and~actions.  
	\item Provided a {\color{myorange}logical reasoning algorithm} in ASP.
	\item Evaluated on 100 problems from WSC and achieved {\color{myorange}100\%} accuracy.

\vspace{0.3cm}
\begin{beamercolorbox}[rounded=true,sep=0.3cm]{expcol}
Extracted knowledge: ``small y prevents y fits". \\
Knowledge type: ``Property prevents Action".\\
\begin{lstlisting}[language = Prolog, style=SC, numbers=right,
numberstyle=\tiny]
%entity y has a trait small
has_k(small,is_trait_of,y). 		
%having trait small prevents the entity to fit another entity
has_k(small,prevents,fits).	 	
%entity y is the recipient of the action fit
has_k(fits,recipient,y).		
\end{lstlisting}
\end{beamercolorbox}
\item Rule 4 {\color{myorange}has no effect} in the reasoning procedure!

\end{itemize}
\vspace{1cm}
%\begin{block} {Winograd Schemas from the Physical Category}
	\textbf{The trophy doesn't fit into the brown suitcase because it's too small.} 
	\vspace{1cm}
	\begin{figure}[width=\columnwidth]
		\centering
		\tikzset{
			treenode/.style = {shape=rectangle, rounded corners,
				draw, align=center,
				top color=white, bottom color=blue!20},
			root/.style     = {treenode, font=\ttfamily\normalsize, bottom color=red!30},
			env/.style      = {treenode, font=\ttfamily\normalsize},
			dummy/.style    = {circle,draw},
			level 1/.style={sibling distance=15cm, level distance = 2.5em},
			level 2/.style={sibling distance=6cm,level distance = 4.5em},
			blueRed/.style={env, top color=blue, bottom color=red} 
		}
		%\begin{document}
		\begin{tikzpicture}
		[
		grow                    = down,
		sibling distance        = 50em,
		level distance          = 5em,
		edge from parent/.style = {draw, -latex},
		every node/.style       = {font=\footnotesize},
		sloped
		]
		\node [root] {is}
		child { node [env, blue] {it}
			child {node [env] {small}
				edge from parent node [below] {trait} }
			edge from parent node [above] {agent}}
		child { node [env, bottom color=red!30] {fit} 
			child {node [env]  {does}
				edge from parent node [above] {support verb} }	
			child {node [env, blue]  {{\small{trophy}}}
				edge from parent node [above] {agent} }
			child {node [env, blue]  {suitcase}
				edge from parent node [above] {recipient}}
			child {node [env]  {not}
				edge from parent node [above] {negation} }
			edge from parent node [above] {cause}};
		
		\end{tikzpicture}
	\end{figure}

%	\textbf{The man couldn't lift his son because he was so weak.} \\
%	\begin{figure}
%		\centering
%		\tikzset{
%			treenode/.style = {shape=rectangle, rounded corners,
%				draw, align=center,
%				top color=white, bottom color=blue!20},
%			root/.style     = {treenode, font=\ttfamily\normalsize, bottom color=red!30},
%			env/.style      = {treenode, font=\ttfamily\normalsize},
%			dummy/.style    = {circle,draw},
%			level 1/.style={sibling distance=15cm, level distance = 2.5em},
%			level 2/.style={sibling distance=6cm,level distance = 4.5em},
%			blueRed/.style={env, top color=blue, bottom color=red} 
%		}
%		%\begin{document}
%		\begin{tikzpicture}
%		[
%		grow                    = down,
%		sibling distance        = 50em,
%		level distance          = 5em,
%		edge from parent/.style = {draw, -latex},
%		every node/.style       = {font=\footnotesize},
%		sloped
%		]
%		\node [root] {was}
%		child { node [env, blue] {he}
%			child {node [env] {weak}
%				edge from parent node [below] {trait} }
%			edge from parent node [above] {agent}}
%		child { node [env, bottom color=red!30] {lift} 
%			child {node [env]  {could}
%				edge from parent node [above] {modal verb} }	
%			child {node [env, blue]  {man}
%				edge from parent node [above] {agent} }
%			child {node [env, blue]  {son}
%				edge from parent node [above] {recipient}}
%			child {node [env]  {not}
%				edge from parent node [above] {negation} }
%			edge from parent node [above] {cause}};
%		\end{tikzpicture}
%	\end{figure}

\begin{block}{Reasoning Algorithm}

 Change of the formalization of the background knowledge \\such that it contributes to the reasoning procedure. \\
 
\begin{beamercolorbox}[rounded=true,sep=0.3cm]{expcol}
 
\begin{lstlisting}[language = Prolog, style =SC, numbers=right,
numberstyle=\tiny]
%entity y is small if we know it could not fit another entity
has_k(small,is_trait_of,y) :- has_k(fit, recipient, y), 
			not has_k(fit,modifier,could).
%entity y should fit another entity 
has_k(fit,recipient,y).
\end{lstlisting}

\end{beamercolorbox}

\begin{itemize}
	\item Rule 1 {\color{myorange}has effect} in the reasoning procedure!
	\item Extend this relation to other problems within one domain (e.g. Physical).
	\item In rule 5, switching \textbf{recipient} with \textbf{agent} leads to no answer!
\end{itemize}

\end{block}
\vspace{-0.6cm}

\begin{block}{Categorization of Winograd Schemas}
\justify 
\begin{itemize}
	\item Inductively analyzed the WSC dataset and identified {\color{myorange}6 categories}.
	\item Categorization based on the {\color{myorange}content} of the Winograd sentence.
	\item Two annotators annotated the entire WSC corpus with these categories.
	\item Calculated Cohen's kappa - measure for inter-rater agreement $\kappa=0.66$
\end{itemize}
 
\vspace{1cm}
\resizebox{\textwidth}{!}{	
	\begin{tabular}{  l | l  }
		\hline
		\textbf{Category}  & \textbf{Example} \\ \hline
		1. Physical &\textbf{S:} The man couldn't lift his son because he is so {\color{myorange}[weak/heavy]}.\\&\textbf{Q:} Who is so [weak/heavy]? \\\hline
		
		2. Emotional &\textbf{S:}  Frank felt {\color{myorange}[vindicated/crushed]} when his longtime rival Bill\\&revealed that he was the winner of the competition.\\&\textbf{Q:} Who was the winner of the competition?\\\hline
		
		3. Interactions &\textbf{S:} Joan made sure to thank Susan for all the help she had {\color{myorange}[given/received]}. \\&\textbf{Q:} Who had [given/received] help?\\ \hline
		
		4. Comparison &\textbf{S:} Joe's uncle can still beat him at tennis, even though he is 30 years {\color{myorange}[older/younger]}.\\&\textbf{Q:} Who is [older/younger]?\\ \hline
		
		5. Causal &\textbf{S:} Pete envies Martin {\color{myorange}[because/although]} he is very successful.\\&\textbf{Q:} Who is very successful? \\ \hline 
		
		6. Multiple knowledge &\textbf{S:}  		
		Sam and Amy are passionately in love, but Amy's parents are unhappy about it,\\& because they are {\color{myorange}[snobs/fifteen]}.  \\&\textbf{Q:} Who are [snobs/fifteen]?\\ \hline
	\end{tabular} } 

\end{block}
\end{block}
\vspace{-1.5cm}
\begin{block}{Conclusions and Outlook}
\begin{itemize}
 \item  Most knowledge-based approaches, so far, have concentrated on the semantic structure of the WS sentence. 
\item None have specified domain specific categories, i.e., the information about the relation between entities and their properties within a certain domain.
\item How to identify the most necessary and the least possible knowledge for solving a WS?
\item An approach, where knowledge is provided only 'by demand' might be more efficient and adequate.
 
\end{itemize}
\end{block}


\end{columns}

\end{frame}
\end{document}
  
